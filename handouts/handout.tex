\documentclass[12pt]{scrartcl}

\setlength{\parindent}{0pt}
\setlength{\parskip}{.25cm}

\usepackage{graphicx}

\usepackage{xcolor}

\definecolor{darkred}{rgb}{0.5,0,0}
\definecolor{darkgreen}{rgb}{0,0.5,0}
\usepackage{hyperref}
\hypersetup{
  letterpaper,
  colorlinks,
  linkcolor=red,
  citecolor=darkgreen,
  menucolor=darkred,
  urlcolor=blue,
  pdfpagemode=none,
  pdftitle={CSCE 156 Lab Handout},
  pdfauthor={Christopher M. Bourke},
  pdfsubject={},
  pdfkeywords={}
}

\definecolor{MyDarkBlue}{rgb}{0,0.08,0.45}
\definecolor{MyDarkRed}{rgb}{0.45,0.08,0}
\definecolor{MyDarkGreen}{rgb}{0.08,0.45,0.08}

\definecolor{mintedBackground}{rgb}{0.95,0.95,0.95}
\definecolor{mintedInlineBackground}{rgb}{.90,.90,1}

%\usepackage{newfloat}
\usepackage[newfloat=true]{minted}
\setminted{mathescape,
               linenos,
               autogobble,
               frame=none,
               framesep=2mm,
               framerule=0.4pt,
               %label=foo,
               xleftmargin=2em,
               xrightmargin=0em,
               startinline=true,  %PHP only, allow it to omit the PHP Tags *** with this option, variables using dollar sign in comments are treated as latex math
               numbersep=10pt, %gap between line numbers and start of line
               style=default, %syntax highlighting style, default is "default"
               			    %gallery: http://help.farbox.com/pygments.html
			    	    %list available: pygmentize -L styles
               bgcolor=mintedBackground} %prevents breaking across pages
               
\setmintedinline{bgcolor={mintedBackground}}
\setminted[text]{bgcolor={mintedBackground},linenos=false,autogobble,xleftmargin=1em}
%\setminted[php]{bgcolor=mintedBackgroundPHP} %startinline=True}
\SetupFloatingEnvironment{listing}{name=Code Sample}
\SetupFloatingEnvironment{listing}{listname=List of Code Samples}

\title{CSCE 156 -- Computer Science II}
\subtitle{Lab 4.0 - Classes \& Constructors}
\author{~}
\date{~}

\begin{document}

\maketitle

\section*{Prior to Lab}

\begin{enumerate}
  \item Review this laboratory handout prior to lab.
  \item Read constructor tutorial: \\
    \url{https://docs.oracle.com/javase/tutorial/java/javaOO/constructors.html}
  \item Read Object Creation tutorial: \\
	\url{http://download.oracle.com/javase/tutorial/java/javaOO/objectcreation.html}
\end{enumerate}

\section*{Lab Objectives \& Topics}
Following the lab, you should be able to:
\begin{itemize}
  \item Use Classes and objects to write Java programs
  \item Understand and use constructors
  \item Understand the visibility of a class's methods and how to use them
\end{itemize}


\section*{Peer Programming Pair-Up}

To encourage collaboration and a team environment, labs will be
structured in a \emph{pair programming} setup.  At the start of
each lab, you will be randomly paired up with another student 
(conflicts such as absences will be dealt with by the lab instructor).
One of you will be designated the \emph{driver} and the other
the \emph{navigator}.  

The navigator will be responsible for reading the instructions and
telling the driver what to do next.  The driver will be in charge of the
keyboard and workstation.  Both driver and navigator are responsible
for suggesting fixes and solutions together.  Neither the navigator
nor the driver is ``in charge.''  Beyond your immediate pairing, you
are encouraged to help and interact and with other pairs in the lab.

Each week you should alternate: if you were a driver last week, 
be a navigator next, etc.  Resolve any issues (you were both drivers
last week) within your pair.  Ask the lab instructor to resolve issues
only when you cannot come to a consensus.  

Because of the peer programming setup of labs, it is absolutely 
essential that you complete any pre-lab activities and familiarize
yourself with the handouts prior to coming to lab.  Failure to do
so will negatively impact your ability to collaborate and work with 
others which may mean that you will not be able to complete the
lab.  

\section*{Getting Started}

Clone the project code for this lab from GitHub in Eclipse using the
URL, \url{https://github.com/cbourke/CSCE156-Lab04}.
Refer to Lab 01 for instructions on how to clone a project from GitHub.

\section*{Classes \& Constructors}

Java is a class-based Object Oriented Programming Language meaning 
that it realizes the concept of objects by allowing you to define 
classes which have member methods and variables.  Instances of classes 
are \emph{instantiated} through a constructor; a method with the same 
name as a class and called using the \mintinline{java}{new} keyword.  
This lab will familiarize you with how classes and their constructors 
are defined and used.  In addition, you will be introduced to some ways 
that Java supports other Object Oriented Principles: Encapsulation and 
Abstraction.

\begin{itemize}
  \item \emph{Encapsulation} is the means by which objects group data 
  	and the methods/functions that act on that data.  It is also the 
	means by which a class's data is protected from the outside 
	world (outside of the object).  Java achieves this by allowing 
	you to define member methods and variables and to specify the 
	visibility of these fields (using the keywords \mintinline{java}{private}, 
	\mintinline{java}{protected}, and \mintinline{java}{public}).
  \item \emph{Abstraction} refers to the means by which an object 
	exposes a public interface to the outside world while hiding 
	the inner workings (the internal representation or the implementation 
	details).  Java's main mechanism for supporting this is the same 
	as with encapsulation though it does provide other means 
	(interfaces, abstract classes). 
  \item \emph{Class Signaling} refers to invoking methods on an instance 
	of a class.  Java uses the dot (or period) operator to signal a class. 
	That is, to invoke one of its methods.
\end{itemize}

\section*{Activities}

We have provided a Java project that simulates a library collections 
system.  It has several classes already defined to model authors, books, 
a library (a collection of books) and a text-based interface which allows 
you to search the collection, add books to the collection, and list the collection.  

\subsection*{Constructing Constructors}

\begin{enumerate}
  \item Run the library program to familiarize yourself with its 
  	functionality.  Note that printing the collection is not fully 
	operational
  \item Complete each of the accessor (getter) methods in the 
	\mintinline{java}{Book} class.  Best Practice Tip: always use 
	the \mintinline{java}{this} keyword to disambiguate the scope 
	of variables and prevent potential problems when subclassing.
  \item Observe that the \mintinline{java}{Book} class does not have 
	a constructor defined.  Examine the \mintinline{java}{addBook()} 
	code and determine how it is possible to create instances of the 
	\mintinline{java}{Book} class without a constructor.
  \item Modify the \mintinline{java}{Book} class by adding and 
	implementing the following constructor.
	
\begin{minted}{java}
public Book(String title, Author author, String isbn, 
            String publishDate) {
  //TODO: your code here
}
\end{minted}

  \item Adding this constructor will cause syntax errors in other 
  	parts of the program.  Think about why and then fix these problems 
	by modifying the code appropriately.
\end{enumerate}  
  
\subsection*{Enforcing Good Encapsulation}

The \mintinline{java}{Book} class is well-designed: it logically groups 
data and methods together that semantically define what a book is and 
how you can use it.  The \mintinline{java}{Author} class is not well 
defined though; its data members are publicly exposed and it has no 
methods at all.

\begin{enumerate}
  \item Redesign the \mintinline{java}{Author} class and make its 
  	member variables private.
  \item Create and use getter methods to make the members accessible 
	to the outside world.  Use these methods where appropriate.
  \item Create setter methods (also called mutator methods) to enable 
	code outside of the \mintinline{java}{Author} class to change the 
	member variables.  Add some data validation: for example, do not 
	allow ``invalid'' values for member variables.  
  \item Add and make use of an appropriate constructor to this class.
  \item Add a method to return a \mintinline{java}{String} that is the 
	author's last name/first name separated by a comma and then utilize 
	it where appropriate (modify the \mintinline{java}{printBooks()} 
	method to use this new method instead of formatting the last 
	name/first name directly).
\end{enumerate}
	
\subsection*{Adding and Using Methods}

The \mintinline{java}{printBooks()} method prints out the title, author, 
and ISBN in a formatted manner one per line.  In this activity, you will 
modify it to output additional information: the year of its publication 
and its age (the number of years since its publication date).  

\begin{enumerate}
  \item Modify the printBooks method in the \mintinline{java}{LibraryDemo} 
    class to output the publication year of each book in another column.  
    Note that the \mintinline{java}{Book} class offers a 
    \mintinline{java}{getPublishDate()} method already implemented for 
    you.
  \item To add the ``age'' column, you will need to add a new method 
    to the \mintinline{java}{Book} class that returns the number of 
    years between the publish date and today.  You may find the following 
    code snippet useful (it utilizes the date/time library provided
    by Java 8):
    
    \mintinline{java}{int years = Period.between(this.publishDate, LocalDate.now()).getYears();}
\end{enumerate}

\subsection*{Advanced Activity (Optional)}


\begin{enumerate}
  \item The \mintinline{java}{printBooks()} method prints books in the 
  	order in which they appear in the list.  This isn't as useful as if 
	they were sorted in some manner.  Read Oracle's tutorial on Object 
	ordering, \url{http://docs.oracle.com/javase/tutorial/collections/interfaces/order.html}.  Write code to use the 
	\mintinline{java}{Collections.sort()} method along with an anonymous
	\mintinline{java}{Comparator<Book>} instance to sort the collection 
	according to the book's title.
  \item Composition is when a class owns instance(s) of other classes 
	(the \mintinline{java}{Book} class owns an instance of the 
	\mintinline{java}{Author} class).  If different instances of a 
	class $A_1$ and $A_2$ are given references to the same object $B$, 
	then changes are made to $B$, the changes will be apparent to both 
	$A_1$ and $A_2$.  Sometimes this behavior is desired.  Other times 
	it is not.  One common method to prevent this is to define a 
	\emph{copy constructor} which takes an instance of the class 
	and performs a \emph{deep copy} of an object by copying each of 
	its fields to the new instance.  For example, the copy constructor 
	for the \mintinline{java}{Author} class may look something like this:
	
\begin{minted}{java}
public Author(Author author) {
  this.firstName = new String(author.firstName);
  this.lastName = new String(author.lastName);
}
\end{minted}
	
	Design and implement a copy constructor for the \mintinline{java}{Book} 
	class and the \mintinline{java}{Library} class.  How do different 
	fields need to be copied?
\end{enumerate}

\end{document}
